% Options for packages loaded elsewhere
\PassOptionsToPackage{unicode}{hyperref}
\PassOptionsToPackage{hyphens}{url}
\PassOptionsToPackage{dvipsnames,svgnames,x11names}{xcolor}
%
\documentclass[
  letterpaper,
  DIV=11,
  numbers=noendperiod]{scrreprt}

\usepackage{amsmath,amssymb}
\usepackage{iftex}
\ifPDFTeX
  \usepackage[T1]{fontenc}
  \usepackage[utf8]{inputenc}
  \usepackage{textcomp} % provide euro and other symbols
\else % if luatex or xetex
  \usepackage{unicode-math}
  \defaultfontfeatures{Scale=MatchLowercase}
  \defaultfontfeatures[\rmfamily]{Ligatures=TeX,Scale=1}
\fi
\usepackage{lmodern}
\ifPDFTeX\else  
    % xetex/luatex font selection
\fi
% Use upquote if available, for straight quotes in verbatim environments
\IfFileExists{upquote.sty}{\usepackage{upquote}}{}
\IfFileExists{microtype.sty}{% use microtype if available
  \usepackage[]{microtype}
  \UseMicrotypeSet[protrusion]{basicmath} % disable protrusion for tt fonts
}{}
\makeatletter
\@ifundefined{KOMAClassName}{% if non-KOMA class
  \IfFileExists{parskip.sty}{%
    \usepackage{parskip}
  }{% else
    \setlength{\parindent}{0pt}
    \setlength{\parskip}{6pt plus 2pt minus 1pt}}
}{% if KOMA class
  \KOMAoptions{parskip=half}}
\makeatother
\usepackage{xcolor}
\setlength{\emergencystretch}{3em} % prevent overfull lines
\setcounter{secnumdepth}{-\maxdimen} % remove section numbering
% Make \paragraph and \subparagraph free-standing
\ifx\paragraph\undefined\else
  \let\oldparagraph\paragraph
  \renewcommand{\paragraph}[1]{\oldparagraph{#1}\mbox{}}
\fi
\ifx\subparagraph\undefined\else
  \let\oldsubparagraph\subparagraph
  \renewcommand{\subparagraph}[1]{\oldsubparagraph{#1}\mbox{}}
\fi

\usepackage{color}
\usepackage{fancyvrb}
\newcommand{\VerbBar}{|}
\newcommand{\VERB}{\Verb[commandchars=\\\{\}]}
\DefineVerbatimEnvironment{Highlighting}{Verbatim}{commandchars=\\\{\}}
% Add ',fontsize=\small' for more characters per line
\usepackage{framed}
\definecolor{shadecolor}{RGB}{241,243,245}
\newenvironment{Shaded}{\begin{snugshade}}{\end{snugshade}}
\newcommand{\AlertTok}[1]{\textcolor[rgb]{0.68,0.00,0.00}{#1}}
\newcommand{\AnnotationTok}[1]{\textcolor[rgb]{0.37,0.37,0.37}{#1}}
\newcommand{\AttributeTok}[1]{\textcolor[rgb]{0.40,0.45,0.13}{#1}}
\newcommand{\BaseNTok}[1]{\textcolor[rgb]{0.68,0.00,0.00}{#1}}
\newcommand{\BuiltInTok}[1]{\textcolor[rgb]{0.00,0.23,0.31}{#1}}
\newcommand{\CharTok}[1]{\textcolor[rgb]{0.13,0.47,0.30}{#1}}
\newcommand{\CommentTok}[1]{\textcolor[rgb]{0.37,0.37,0.37}{#1}}
\newcommand{\CommentVarTok}[1]{\textcolor[rgb]{0.37,0.37,0.37}{\textit{#1}}}
\newcommand{\ConstantTok}[1]{\textcolor[rgb]{0.56,0.35,0.01}{#1}}
\newcommand{\ControlFlowTok}[1]{\textcolor[rgb]{0.00,0.23,0.31}{#1}}
\newcommand{\DataTypeTok}[1]{\textcolor[rgb]{0.68,0.00,0.00}{#1}}
\newcommand{\DecValTok}[1]{\textcolor[rgb]{0.68,0.00,0.00}{#1}}
\newcommand{\DocumentationTok}[1]{\textcolor[rgb]{0.37,0.37,0.37}{\textit{#1}}}
\newcommand{\ErrorTok}[1]{\textcolor[rgb]{0.68,0.00,0.00}{#1}}
\newcommand{\ExtensionTok}[1]{\textcolor[rgb]{0.00,0.23,0.31}{#1}}
\newcommand{\FloatTok}[1]{\textcolor[rgb]{0.68,0.00,0.00}{#1}}
\newcommand{\FunctionTok}[1]{\textcolor[rgb]{0.28,0.35,0.67}{#1}}
\newcommand{\ImportTok}[1]{\textcolor[rgb]{0.00,0.46,0.62}{#1}}
\newcommand{\InformationTok}[1]{\textcolor[rgb]{0.37,0.37,0.37}{#1}}
\newcommand{\KeywordTok}[1]{\textcolor[rgb]{0.00,0.23,0.31}{#1}}
\newcommand{\NormalTok}[1]{\textcolor[rgb]{0.00,0.23,0.31}{#1}}
\newcommand{\OperatorTok}[1]{\textcolor[rgb]{0.37,0.37,0.37}{#1}}
\newcommand{\OtherTok}[1]{\textcolor[rgb]{0.00,0.23,0.31}{#1}}
\newcommand{\PreprocessorTok}[1]{\textcolor[rgb]{0.68,0.00,0.00}{#1}}
\newcommand{\RegionMarkerTok}[1]{\textcolor[rgb]{0.00,0.23,0.31}{#1}}
\newcommand{\SpecialCharTok}[1]{\textcolor[rgb]{0.37,0.37,0.37}{#1}}
\newcommand{\SpecialStringTok}[1]{\textcolor[rgb]{0.13,0.47,0.30}{#1}}
\newcommand{\StringTok}[1]{\textcolor[rgb]{0.13,0.47,0.30}{#1}}
\newcommand{\VariableTok}[1]{\textcolor[rgb]{0.07,0.07,0.07}{#1}}
\newcommand{\VerbatimStringTok}[1]{\textcolor[rgb]{0.13,0.47,0.30}{#1}}
\newcommand{\WarningTok}[1]{\textcolor[rgb]{0.37,0.37,0.37}{\textit{#1}}}

\providecommand{\tightlist}{%
  \setlength{\itemsep}{0pt}\setlength{\parskip}{0pt}}\usepackage{longtable,booktabs,array}
\usepackage{calc} % for calculating minipage widths
% Correct order of tables after \paragraph or \subparagraph
\usepackage{etoolbox}
\makeatletter
\patchcmd\longtable{\par}{\if@noskipsec\mbox{}\fi\par}{}{}
\makeatother
% Allow footnotes in longtable head/foot
\IfFileExists{footnotehyper.sty}{\usepackage{footnotehyper}}{\usepackage{footnote}}
\makesavenoteenv{longtable}
\usepackage{graphicx}
\makeatletter
\def\maxwidth{\ifdim\Gin@nat@width>\linewidth\linewidth\else\Gin@nat@width\fi}
\def\maxheight{\ifdim\Gin@nat@height>\textheight\textheight\else\Gin@nat@height\fi}
\makeatother
% Scale images if necessary, so that they will not overflow the page
% margins by default, and it is still possible to overwrite the defaults
% using explicit options in \includegraphics[width, height, ...]{}
\setkeys{Gin}{width=\maxwidth,height=\maxheight,keepaspectratio}
% Set default figure placement to htbp
\makeatletter
\def\fps@figure{htbp}
\makeatother

\KOMAoption{captions}{tableheading}
\makeatletter
\makeatother
\makeatletter
\makeatother
\makeatletter
\@ifpackageloaded{caption}{}{\usepackage{caption}}
\AtBeginDocument{%
\ifdefined\contentsname
  \renewcommand*\contentsname{Table of contents}
\else
  \newcommand\contentsname{Table of contents}
\fi
\ifdefined\listfigurename
  \renewcommand*\listfigurename{List of Figures}
\else
  \newcommand\listfigurename{List of Figures}
\fi
\ifdefined\listtablename
  \renewcommand*\listtablename{List of Tables}
\else
  \newcommand\listtablename{List of Tables}
\fi
\ifdefined\figurename
  \renewcommand*\figurename{Figure}
\else
  \newcommand\figurename{Figure}
\fi
\ifdefined\tablename
  \renewcommand*\tablename{Table}
\else
  \newcommand\tablename{Table}
\fi
}
\@ifpackageloaded{float}{}{\usepackage{float}}
\floatstyle{ruled}
\@ifundefined{c@chapter}{\newfloat{codelisting}{h}{lop}}{\newfloat{codelisting}{h}{lop}[chapter]}
\floatname{codelisting}{Listing}
\newcommand*\listoflistings{\listof{codelisting}{List of Listings}}
\makeatother
\makeatletter
\@ifpackageloaded{caption}{}{\usepackage{caption}}
\@ifpackageloaded{subcaption}{}{\usepackage{subcaption}}
\makeatother
\makeatletter
\@ifpackageloaded{tcolorbox}{}{\usepackage[skins,breakable]{tcolorbox}}
\makeatother
\makeatletter
\@ifundefined{shadecolor}{\definecolor{shadecolor}{rgb}{.97, .97, .97}}
\makeatother
\makeatletter
\makeatother
\makeatletter
\makeatother
\ifLuaTeX
  \usepackage{selnolig}  % disable illegal ligatures
\fi
\IfFileExists{bookmark.sty}{\usepackage{bookmark}}{\usepackage{hyperref}}
\IfFileExists{xurl.sty}{\usepackage{xurl}}{} % add URL line breaks if available
\urlstyle{same} % disable monospaced font for URLs
\hypersetup{
  colorlinks=true,
  linkcolor={blue},
  filecolor={Maroon},
  citecolor={Blue},
  urlcolor={Blue},
  pdfcreator={LaTeX via pandoc}}

\author{}
\date{}

\begin{document}
\ifdefined\Shaded\renewenvironment{Shaded}{\begin{tcolorbox}[sharp corners, frame hidden, interior hidden, breakable, borderline west={3pt}{0pt}{shadecolor}, boxrule=0pt, enhanced]}{\end{tcolorbox}}\fi

\hypertarget{o-modelo-linear-dinuxe2mico}{%
\chapter{O modelo linear dinâmico}\label{o-modelo-linear-dinuxe2mico}}

\hypertarget{o-modelo-linear-dinuxe2mico-1}{%
\section{O modelo linear dinâmico}\label{o-modelo-linear-dinuxe2mico-1}}

Seja \(y_1,\ldots,t\) uma série temporal. Seja \(D_j={y_1,\ldots,y_j}\).
Dizemos que \(y_t\) é um modelo linear dinâmico se

\[\begin{align}
y_t|\boldsymbol{\theta}_t,D_{t-1}&\sim\hbox{Normal}(\boldsymbol{F}_t'\boldsymbol{\theta}_t,V_t)\\
\boldsymbol{\theta}_t|\boldsymbol{\theta}_{t-1},D_{t-1}&\sim\hbox{Normal}(\boldsymbol{G}_t\boldsymbol{\theta}_{t-1},\boldsymbol{W}_t)
\end{align}\]

A expressão acima, os \(\theta\)'s são denominados estados. Para a
completa especificação do modelo, devemos informar valores iniciais
\(m_0\) e \(C_0\) que representam nossa opinião sobre os estados antes
do tempo \(1\):

\[\theta_0\sim\ N(m_0,C_0)\] Escolhas diferentes para
\(\boldsymbol{F}_t\) e \(\boldsymbol{G}_t\) permitem acomodar sinais
diferentes.

Pode-se mostrar que \(y_{t+h}|D_t\) tem distribuição normal. Como
\(t+h\) é um tempo não observado, essa é a distribuição para previsões.
Neste caso, a função de previsão para o horizonte \(h\) é

\[f_t(h)=E(Y_{t+h}|D_{t})\] onde \(E(.)\) sempre deve ser lido como
\textbf{média}.

O pacote para lidar com modelos lineares dinâmicos é o \texttt{dlm}. A
função abaixo é utilizada para estimar as variâncias desconhecidas do
modelo e deve sempre ser colocada no \emph{environment} do \texttt{R}:

\begin{Shaded}
\begin{Highlighting}[]
\FunctionTok{require}\NormalTok{(dlm)}
\end{Highlighting}
\end{Shaded}

\begin{verbatim}
Carregando pacotes exigidos: dlm
\end{verbatim}

\begin{verbatim}
Warning: package 'dlm' was built under R version 4.3.1
\end{verbatim}

\begin{Shaded}
\begin{Highlighting}[]
\NormalTok{modFim }\OtherTok{\textless{}{-}} \ControlFlowTok{function}\NormalTok{(y,mod)\{}
\NormalTok{  ffbs }\OtherTok{\textless{}{-}} \FunctionTok{dlmGibbsDIG}\NormalTok{(y, }\AttributeTok{mod =}\NormalTok{ mod, }\AttributeTok{n.sample =} \DecValTok{5000}\NormalTok{,}
                    \AttributeTok{a.y=}\DecValTok{1}\NormalTok{,}\AttributeTok{b.y=}\DecValTok{100}\NormalTok{,}\AttributeTok{a.theta=}\DecValTok{1}\NormalTok{,}\AttributeTok{b.theta=}\DecValTok{100}\NormalTok{,}
                    \AttributeTok{save.states =} \ConstantTok{FALSE}\NormalTok{, }\AttributeTok{thin =} \DecValTok{0}\NormalTok{)}

\NormalTok{v\_sim  }\OtherTok{\textless{}{-}} \FunctionTok{sample}\NormalTok{(ffbs}\SpecialCharTok{$}\NormalTok{dV[}\SpecialCharTok{{-}}\NormalTok{(}\DecValTok{1}\SpecialCharTok{:}\DecValTok{2500}\NormalTok{)],}\DecValTok{2500}\NormalTok{,T)}

\NormalTok{q }\OtherTok{\textless{}{-}} \FunctionTok{dim}\NormalTok{(ffbs}\SpecialCharTok{$}\NormalTok{dW)[}\DecValTok{2}\NormalTok{]}
\NormalTok{w\_sim }\OtherTok{\textless{}{-}} \ConstantTok{NULL}
\ControlFlowTok{for}\NormalTok{(j }\ControlFlowTok{in} \DecValTok{1}\SpecialCharTok{:}\NormalTok{q)\{}
\NormalTok{ w\_sim }\OtherTok{\textless{}{-}} \FunctionTok{c}\NormalTok{(w\_sim, }\FunctionTok{mean}\NormalTok{(}\FunctionTok{sample}\NormalTok{(ffbs}\SpecialCharTok{$}\NormalTok{dW[,j][}\SpecialCharTok{{-}}\NormalTok{(}\DecValTok{1}\SpecialCharTok{:}\DecValTok{2500}\NormalTok{)],}\DecValTok{2500}\NormalTok{,T)))}
\NormalTok{\}}
\CommentTok{\# declarando as variâncias na quádrupla}
\NormalTok{mod}\SpecialCharTok{$}\NormalTok{V }\OtherTok{\textless{}{-}} \FunctionTok{mean}\NormalTok{(v\_sim)}
\NormalTok{mod}\SpecialCharTok{$}\NormalTok{W }\OtherTok{\textless{}{-}} \FunctionTok{diag}\NormalTok{( w\_sim)}
\FunctionTok{return}\NormalTok{(mod)}
\NormalTok{\}}
\end{Highlighting}
\end{Shaded}

Também pode-se mostrar que \(\theta_{t-h}|D_t\) tem distribuição normal.
Como se trata da distribuição dos estados após verificar toda a série
temporal, esta é a distribuição para a suavização.



\end{document}
